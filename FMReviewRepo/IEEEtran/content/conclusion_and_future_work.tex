\section{Conclusion and Future Work}

Focusing on the efficiency of review meetings we covered an essential part regarding the document review meetings and whether they have an influence on enhancing code quality.  


As we only focus on document review meetings a possible future work is to conduct a similar experiment study taking a closer look at code review meetings. The results can further be compared to studys, where only commonly used code inspection techiques are used, such as static or dynamic code inspections, and draw conclusions whether it is usefull to aditionally conduct code review meetings. Further it can be looked on possible correlations, e.g. between the experience on conducting reviews and the programming experience of the participants. Measuring the duration of the review meetings and observing when then most findings are detected can give additional information espacially on how long a session should last to be efficient.  


To have a better understanding of how many costs are involved within a review meeting it is also important to cunduct a study to detect e.g. how much a bug costs during the various stages of software developement, as well as how much it costs to engage a developer in this process etc. With this knowledge a better understanding of the importance of review meetings can be achieved.     