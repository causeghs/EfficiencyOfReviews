\section{Introduction}
Traditional code and document inspection in terms of discovery and comprehension of mistakes and bugs is one of the most important processes in software development.
In addition to the intuitive approach of manually discovering mistakes, side effects and bugs in software and documents, there are various techniques to further enhance code and documentation quality, ranging from spell-checking, static and dynamic code analyses, automatic code quality monitoring to project check-in policies.
These techniques have one thing in common. They can mostly be applied (semi-)automatically without the need of interacting with other fellow developers.

Another technique for improving code quality, that utilizes the ‘\textit{many eyes see better than two}’ principle are review meetings.
These developer meetings generally take place after each developer independently reviewed the code or document in question.
The idea of these meetings is that all independently discovered issues get accumulated, presented and discussed in a public manner. 
Earlier work on code review meetings has shown that such review meetings are useful for discovering potential bugs, improving code quality and enforcing standards (software reviews: the state of the practice).
The question that yet remains to be answered is if code review meetings are an efficient way of improving code quality in general.