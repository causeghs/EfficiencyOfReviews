\section{Introduction}
Traditional code inspection in terms of discovery and comprehension of bugs is one of the most important processes in software development.
In addition to the intuitive approach of manually discovering side effects and bugs in software, there are various techniques to further enhance code quality, ranging from static and dynamic code analyses over automatic code quality monitoring to project check-in policies.
These techniques have one thing in common. They can mostly be applied (semi-)automatically without the need of interacting with other fellow developers.
Another technique for improving code quality, that utilizes the ‘many eyes see better than two’ principle are code review meetings.
These developer meetings generally take place after each developer reviewed the code independently.
The idea of these meetings is that all independently accumulated results get presented and discussed in a public manner. 
Earlier work on code review meetings has shown that such review meetings are useful for discovering potential bugs, improving code quality and enforcing standards (software reviews: the state of the practice).
The question that yet remains to be answered is if code review meetings are an efficient way of improving code quality in general.