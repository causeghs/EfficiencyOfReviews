\section{Threats to Validity}

\subsection{Conclusion Validity}

The threat level to conclusion validity of our experiment is rather high. This is due to the fact
that some of our measurements could not be automatically calculated but instead had to be determined
manually by the conductors. In order to reduce this threat, we decided to let three conductors
determine the values independently and finally took the median as our final result.
Also since the specification was created by us, we are biased when evaluating the findings. This threat can be eliminated by separating the persons doing the specification and the persons evaluating the findings.
Our calculations on the efficiency hypothesis H2 are mainly based on money estimates, and the costs of developers varies between countries, companies and individual position within a company.

\subsection{Internal Validity}

Due to the fact that the participants had to be informed about the experiment to give us their
consent, they knew that some part of their work will be evaluated and thus might have been
influenced in their behavior by the experiment, e.g. resulting in them making more efforts than they would have done otherwise.

Since we cannot prevent students from communicating with each other, it is unavoidable that information about the specification or experiment in general could be exchanged.

\subsection{Construct Validity}

The construct validity between the treatment and the cause construct is a given, since they are identical, being the review meeting.

As for the validity between the outcome and the construct of the effect, ?

\subsection{External Validity}

Our study also contains external threats to validity. Since every group of students in the "SoPra"
has to solve the exact same task, all produces code or specifications should ideally be semantically
equal and thus our results might not necessarily be generalizable to all kind of software systems.

Another external factor that threatens validity are the participants of the study. These are limited
to students of the university Stuttgart and thus represent only a minor part of software engineers.

Lastly, the size of the "SoPra" is very small, since it is only a six month project for three developers and thus is far off from realistic industrial projects.

