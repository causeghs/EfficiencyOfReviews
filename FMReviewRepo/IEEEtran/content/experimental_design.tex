\section{Experimental Design}

\subsection{Research Questions}

\textit{RQ1:} How effective are review meetings in regards to improving document or code quality?

\textit{RQ2:} Are code review meetings an efficient way of increasing document or code quality?

%\textit{RQ3:} Which requirements have to be met in order to have effective %review meetings?

%\textit{RQ4:} Which requirements have to be met in order to have efficient %review meetings?

\subsection{Hypothesis}

H1: Review meetings are effective at improving document or code quality.

H10: Review meetings are not effective at improving document or code quality.

H2: Review meetings are an efficient way of increasing document or code quality.

H20: Review meetings are not an efficient way of increasing document or code quality.

\subsection{Design}

Quantitative and qualitative analysis.?
One factor 2 treatment.

For our quantitative analysis we used a one factor two treatment design. The first treatment is the control treatment of dong no review meetings, while the second treatment is the execution of a review meeting. Since doing no review does not influence the subjects, each group was assigned to both treatments.

\definecolor{heading}{RGB}{200,200,255}
\definecolor{a}{RGB}{220,220,255}
\definecolor{b}{RGB}{230,230,255}

\begin{table}
\begin{tabular}{lcr}
  \rowcolor{heading}Subjects & No Review Meeting & Review Meeting \\
  \rowcolor{a}"SoPra" Group 1 & x & x \\
  \rowcolor{b}"SoPra" Group 2 & x & x \\
  \rowcolor{a}... & x & x \\
\end{tabular}
\caption{The one factor two treatment design used for our experiment.}
\end{table}


\subsection{Objects}

The idea of our experiment was that we analyse the report lists that result out of the review meetings and compare them to the merged lists of reports from each individual reviewer.

\subsection{Data collection Procedure}

\subsection{Analysis Procedure}

\subsection{Validity Procedure}