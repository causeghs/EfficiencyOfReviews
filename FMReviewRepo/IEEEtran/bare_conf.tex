
%% bare_conf.tex
%% V1.4b
%% 2015/08/26
%% by Michael Shell
%% See:
%% http://www.michaelshell.org/
%% for current contact information.
%%
%% This is a skeleton file demonstrating the use of IEEEtran.cls
%% (requires IEEEtran.cls version 1.8b or later) with an IEEE
%% conference paper.
%%
%% Support sites:
%% http://www.michaelshell.org/tex/ieeetran/
%% http://www.ctan.org/pkg/ieeetran
%% and
%% http://www.ieee.org/

%%*************************************************************************
%% Legal Notice:
%% This code is offered as-is without any warranty either expressed or
%% implied; without even the implied warranty of MERCHANTABILITY or
%% FITNESS FOR A PARTICULAR PURPOSE! 
%% User assumes all risk.
%% In no event shall the IEEE or any contributor to this code be liable for
%% any damages or losses, including, but not limited to, incidental,
%% consequential, or any other damages, resulting from the use or misuse
%% of any information contained here.
%%
%% All comments are the opinions of their respective authors and are not
%% necessarily endorsed by the IEEE.
%%
%% This work is distributed under the LaTeX Project Public License (LPPL)
%% ( http://www.latex-project.org/ ) version 1.3, and may be freely used,
%% distributed and modified. A copy of the LPPL, version 1.3, is included
%% in the base LaTeX documentation of all distributions of LaTeX released
%% 2003/12/01 or later.
%% Retain all contribution notices and credits.
%% ** Modified files should be clearly indicated as such, including  **
%% ** renaming them and changing author support contact information. **
%%*************************************************************************


% *** Authors should verify (and, if needed, correct) their LaTeX system  ***
% *** with the testflow diagnostic prior to trusting their LaTeX platform ***
% *** with production work. The IEEE's font choices and paper sizes can   ***
% *** trigger bugs that do not appear when using other class files.       ***                          ***
% The testflow support page is at:
% http://www.michaelshell.org/tex/testflow/



\documentclass[conference]{IEEEtran}
% Some Computer Society conferences also require the compsoc mode option,
% but others use the standard conference format.
%
% If IEEEtran.cls has not been installed into the LaTeX system files,
% manually specify the path to it like:
% \documentclass[conference]{../sty/IEEEtran}





% Some very useful LaTeX packages include:
% (uncomment the ones you want to load)


% *** MISC UTILITY PACKAGES ***
%
%\usepackage{ifpdf}
% Heiko Oberdiek's ifpdf.sty is very useful if you need conditional
% compilation based on whether the output is pdf or dvi.
% usage:
% \ifpdf
%   % pdf code
% \else
%   % dvi code
% \fi
% The latest version of ifpdf.sty can be obtained from:
% http://www.ctan.org/pkg/ifpdf
% Also, note that IEEEtran.cls V1.7 and later provides a builtin
% \ifCLASSINFOpdf conditional that works the same way.
% When switching from latex to pdflatex and vice-versa, the compiler may
% have to be run twice to clear warning/error messages.

\usepackage{colortbl}
\usepackage[dvipsnames]{xcolor}
\usepackage{lipsum}
\usepackage{stfloats}
\usepackage{array}
\usepackage{booktabs}
\usepackage{url}

% *** CITATION PACKAGES ***
%
%\usepackage{cite}
% cite.sty was written by Donald Arseneau
% V1.6 and later of IEEEtran pre-defines the format of the cite.sty package
% \cite{} output to follow that of the IEEE. Loading the cite package will
% result in citation numbers being automatically sorted and properly
% "compressed/ranged". e.g., [1], [9], [2], [7], [5], [6] without using
% cite.sty will become [1], [2], [5]--[7], [9] using cite.sty. cite.sty's
% \cite will automatically add leading space, if needed. Use cite.sty's
% noadjust option (cite.sty V3.8 and later) if you want to turn this off
% such as if a citation ever needs to be enclosed in parenthesis.
% cite.sty is already installed on most LaTeX systems. Be sure and use
% version 5.07-10 (2009-03-20) and later if using hyperref.sty.
% The latest version can be obtained at:
% http://www.ctan.org/pkg/cite
% The documentation is contained in the cite.sty file itself.






% *** GRAPHICS RELATED PACKAGES ***
%
\ifCLASSINFOpdf
  % \usepackage[pdftex]{graphicx}
  % declare the path(s) where your graphic files are
  % \graphicspath{{../pdf/}{../jpeg/}}
  % and their extensions so you won't have to specify these with
  % every instance of \includegraphics
  % \DeclareGraphicsExtensions{.pdf,.jpeg,.png}
\else
  % or other class option (dvipsone, dvipdf, if not using dvips). graphicx
  % will default to the driver specified in the system graphics.cfg if no
  % driver is specified.
  % \usepackage[dvips]{graphicx}
  % declare the path(s) where your graphic files are
  % \graphicspath{{../eps/}}
  % and their extensions so you won't have to specify these with
  % every instance of \includegraphics
  % \DeclareGraphicsExtensions{.eps}
\fi
% graphicx was written by David Carlisle and Sebastian Rahtz. It is
% required if you want graphics, photos, etc. graphicx.sty is already
% installed on most LaTeX systems. The latest version and documentation
% can be obtained at: 
% http://www.7-10ctan.org/pkg/graphicx
% Another good source of documentation is "Using Imported Graphics in
% LaTeX2e" by Keith Reckdahl which can be found at:
% http://www.ctan.org/pkg/epslatex
%
% latex, and pdflatex in dvi mode, support graphics in encapsulated
% postscript (.eps) format. pdflatex in pdf mode supports graphics
% in .pdf, .jpeg, .png and .mps (metapost) formats. Users should ensure
% that all non-photo figures use a vector format (.eps, .pdf, .mps) and
% not a bitmapped formats (.jpeg, .png). The IEEE frowns on bitmapped formats
% which can result in "jaggedy"/blurry rendering of lines and letters as
% well as large increases in file sizes.
%
% You can find documentation about the pdfTeX application at:
% http://www.tug.org/applications/pdftex





% *** MATH PACKAGES ***
%
%\usepackage{amsmath}
% A popular package from the American Mathematical Society that provides
% many useful and powerful commands for dealing with mathematics.
%
% Note that the amsmath package sets \interdisplaylinepenalty to 10000
% thus preventing page breaks from occurring within multiline equations. Use:
%\interdisplaylinepenalty=2500
% after loading amsmath to restore such page breaks as IEEEtran.cls normally
% does. amsmath.sty is already installed on most LaTeX systems. The latest
% version and documentation can be obtained at:
% http://www.ctan.org/pkg/amsmath





% *** SPECIALIZED LIST PACKAGES ***
%
%\usepackage{algorithmic}
% algorithmic.sty was written by Peter Williams and Rogerio Brito.
% This package provides an algorithmic environment fo describing algorithms.
% You can use the algorithmic environment in-text or within a figure
% environment to provide for a floating algorithm. Do NOT use the algorithm
% floating environment provided by algorithm.sty (by the same authors) or
% algorithm2e.sty (by Christophe Fiorio) as the IEEE does not use dedicated
% algorithm float types and packages that provide these will not provide
% correct IEEE style captions. The latest version and documentation of
% algorithmic.sty can be obtained at:
% http://www.ctan.org/pkg/algorithms
% Also of interest may be the (relatively newer and more customizable)
% algorithmicx.sty package by Szasz Janos:
% http://www.ctan.org/pkg/algorithmicx




% *** ALIGNMENT PACKAGES ***
%
%\usepackage{array}
% Frank Mittelbach's and David Carlisle's array.sty patches and improves
% the standard LaTeX2e array and tabular environments to provide better
% appearance and additional user controls. As the default LaTeX2e table
% generation code is lacking to the point of almost being broken with
% respect to the quality of the end results, all users are strongly
% advised to use an enhanced (at the very least that provided by array.sty)
% set of table tools. array.sty is already installed on most systems. The
% latest version and documentation can be obtained at:
% http://www.ctan.org/pkg/array


% IEEEtran contains the IEEEeqnarray family of commands that can be used to
% generate multiline equations as well as matrices, tables, etc., of high
% quality.




% *** SUBFIGURE PACKAGES ***
%\ifCLASSOPTIONcompsoc
%  \usepackage[caption=false,font=normalsize,labelfont=sf,textfont=sf]{subfig}
%\else
%  \usepackage[caption=false,font=footnotesize]{subfig}
%\fi
% subfig.sty, written by Steven Douglas Cochran, is the modern replacement
% for subfigure.sty, the latter of which is no longer maintained and is
% incompatible with some LaTeX packages including fixltx2e. However,
% subfig.sty requires and automatically loads Axel Sommerfeldt's caption.sty
% which will override IEEEtran.cls' handling of captions and this will result
% in non-IEEE style figure/table captions. To prevent this problem, be sure
% and invoke subfig.sty's "caption=false" package option (available since
% subfig.sty version 1.3, 2005/06/28) as this is will preserve IEEEtran.cls
% handling of captions.
% Note that the Computer Society format requires a larger sans serif font
% than the serif footnote size font used in traditional IEEE formatting
% and thus the need to invoke different subfig.sty package options depending
% on whether compsoc mode has been enabled.
%
% The latest version and documentation of subfig.sty can be obtained at:
% http://www.ctan.org/pkg/subfig




% *** FLOAT PACKAGES ***
%
%\usepackage{fixltx2e}
% fixltx2e, the successor to the earlier fix2col.sty, was written by
% Frank Mittelbach and David Carlisle. This package corrects a few problems
% in the LaTeX2e kernel, the most notable of which is that in current
% LaTeX2e releases, the ordering of single and double column floats is not
% guaranteed to be preserved. Thus, an unpatched LaTeX2e can allow a
% single column figure to be placed prior to an earlier double column
% figure.
% Be aware that LaTeX2e kernels dated 2015 and later have fixltx2e.sty's
% corrections already built into the system in which case a warning will
% be issued if an attempt is made to load fixltx2e.sty as it is no longer
% needed.
% The latest version and documentation can be found at:
% http://www.ctan.org/pkg/fixltx2e


%\usepackage{stfloats}
% stfloats.sty was written by Sigitas Tolusis. This package gives LaTeX2e
% the ability to do double column floats at the bottom of the page as well
% as the top. (e.g., "\begin{figure*}[!b]" is not normally possible in
% LaTeX2e). It also provides a command:
%\fnbelowfloat
% to enable the placement of footnotes below bottom floats (the standard
% LaTeX2e kernel puts them above bottom floats). This is an invasive package
% which rewrites many portions of the LaTeX2e float routines. It may not work
% with other packages that modify the LaTeX2e float routines. The latest
% version and documentation can be obtained at:
% http://www.ctan.org/pkg/stfloats
% Do not use the stfloats baselinefloat ability as the IEEE does not allow
% \baselineskip to stretch. Authors submitting work to the IEEE should note
% that the IEEE rarely uses double column equations and that authors should try
% to avoid such use. Do not be tempted to use the cuted.sty or midfloat.sty
% packages (also by Sigitas Tolusis) as the IEEE does not format its papers in
% such ways.
% Do not attempt to use stfloats with fixltx2e as they are incompatible.
% Instead, use Morten Hogholm'a dblfloatfix which combines the features
% of both fixltx2e and stfloats:
%
% \usepackage{dblfloatfix}
% The latest version can be found at:
% http://www.ctan.org/pkg/dblfloatfix




% *** PDF, URL AND HYPERLINK PACKAGES ***
%
%\usepackage{url}
% url.sty was written by Donald Arseneau. It provides better support for
% handling and breaking URLs. url.sty is already installed on most LaTeX
% systems. The latest version and documentation can be obtained at:
% http://www.ctan.org/pkg/url
% Basically, \url{my_url_here}.




% *** Do not adjust lengths that control margins, column widths, etc. ***
% *** Do not use packages that alter fonts (such as pslatex).         ***
% There should be no need to do such things with IEEEtran.cls V1.6 and later.
% (Unless specifically asked to do so by the journal or conference you plan
% to submit to, of course. )


% correct bad hyphenation here
\hyphenation{op-tical net-works semi-conduc-tor}


\begin{document}
%
% paper title
% Titles are generally capitalized except for words such as a, an, and, as,
% at, but, by, for, in, nor, of, on, or, the, to and up, which are usually
% not capitalized unless they are the first or last word of the title.
% Linebreaks \\ can be used within to get better formatting as desired.
% Do not put math or special symbols in the title.
\title{Efficiency of Review Meetings}


% author names and affiliations
% use a multiple column layout for up to three different
% affiliations

\author{\IEEEauthorblockN{Haris Causegic,
Aretina Iazzolino,
Andreas Korge,
Josip Ledic and
Thommy Zelenik}
\IEEEauthorblockA{University of Stuttgart\\
Institute of Software Technology\\
Stuttgart, Germany\\
Email: inf82831@stud.uni-stuttgart.de}}


% conference papers do not typically use \thanks and this command
% is locked out in conference mode. If really needed, such as for
% the acknowledgment of grants, issue a \IEEEoverridecommandlockouts
% after \documentclass

% for over three affiliations, or if they all won't fit within the width
% of the page, use this alternative format:
% 
%\author{\IEEEauthorblockN{Michael Shell\IEEEauthorrefmark{1},
%Homer Simpson\IEEEauthorrefmark{2},
%James Kirk\IEEEauthorrefmark{3}, 
%Montgomery Scott\IEEEauthorrefmark{3} and
%Eldon Tyrell\IEEEauthorrefmark{4}}
%\IEEEauthorblockA{\IEEEauthorrefmark{1}School of Electrical and Computer Engineering\\
%Georgia Institute of Technology,
%Atlanta, Georgia 30332--0250\\ Email: see http://www.michaelshell.org/contact.html}
%\IEEEauthorblockA{\IEEEauthorrefmark{2}Twentieth Century Fox, Springfield, USA\\
%Email: homer@thesimpsons.com}
%\IEEEauthorblockA{\IEEEauthorrefmark{3}Starfleet Academy, San Francisco, California 96678-2391\\
%Telephone: (800) 555--1212, Fax: (888) 555--1212}
%\IEEEauthorblockA{\IEEEauthorrefmark{4}Tyrell Inc., 123 Replicant Street, Los Angeles, California 90210--4321}}




% use for special paper notices
%\IEEEspecialpapernotice{(Invited Paper)}




% make the title area
\maketitle

% As a general rule, do not put math, special symbols or citations
% in the abstract
\begin{abstract}
\textit{Background:} Review meetings are a recognized method for enhancing code and document quality in software development.
It is widely accepted that these meetings help discovering potential bugs and problems in software.
Nonetheless, review meetings involve considerable expenses and therefore it is reasonable to investigate if the meetings themselves do have a substantial impact on the number of findings or if meeting-less code inspections are equally effective.

\textit{Research questions:} \textit{(1)} Are review meetings effective in regards to improving document or code quality? \textit{(2)} How efficient are code review meetings as a means to increasing document or code quality? \textit{(3)} Which positive effects besides improving code quality do review meetings entail?

\textit{Methodology:} We designed an experiment where we compared the finding lists before and after a review meeting. The review meetings were conducted by student teams of the "Software Praktikum" at the University of Stuttgart. The subject of the reviews was a software specification. We conducted semi-structured interviews with the participants and we observed the meetings passively.

\textit{Results:} \textit{--no real results, because the experiment wasn't conducted yet--}

\textit{Conclusion:} \textit{--no conclusion, because the experiment wasn't conducted yet--}

\end{abstract}

% no keywords




% For peer review papers, you can put extra information on the cover
% page as needed:
% \ifCLASSOPTIONpeerreview
% \begin{center} \bfseries EDICS Category: 3-BBND \end{center}
% \fi
%
% For peerreview papers, this IEEEtran command inserts a page break and
% creates the second title. It will be ignored for other modes.
\IEEEpeerreviewmaketitle



%\section{Introduction}
% no \IEEEPARstart
%This demo file is intended to serve as a ``starter file''
%for IEEE conference papers produced under \LaTeX\ using
%IEEEtran.cls version 1.8b and later.
% You must have at least 2 lines in the paragraph with the drop letter
% (should never be an issue)
%I wish you the best of success.

%\hfill mds
 
%\hfill August 26, 2015

%\subsection{Subsection Heading Here}
%Subsection text here.


%\subsubsection{Subsubsection Heading Here}
%Subsubsection text here.


% An example of a floating figure using the graphicx package.
% Note that \label must occur AFTER (or within) \caption.
% For figures, \caption should occur after the \includegraphics.
% Note that IEEEtran v1.7 and later has special internal code that
% is designed to preserve the operation of \label within \caption
% even when the captionsoff option is in effect. However, because
% of issues like this, it may be the safest practice to put all your
% \label just after \caption rather than within \caption{}.
%
% Reminder: the "draftcls" or "draftclsnofoot", not "draft", class
% option should be used if it is desired that the figures are to be
% displayed while in draft mode.
%
%\begin{figure}[!t]
%\centering
%\includegraphics[width=2.5in]{myfigure}
% where an .eps filename suffix will be assumed under latex, 
% and a .pdf suffix will be assumed for pdflatex; or what has been declared
% via \DeclareGraphicsExtensions.
%\caption{Simulation results for the network.}
%\label{fig_sim}
%\end{figure}

% Note that the IEEE typically puts floats only at the top, even when this
% results in a large percentage of a column being occupied by floats.


% An example of a double column floating figure using two subfigures.
% (The subfig.sty package must be loaded for this to work.)
% The subfigure \label commands are set within each subfloat command,
% and the \label for the overall figure must come after \caption.
% \hfil is used as a separator to get equal spacing.
% Watch out that the combined width of all the subfigures on a 
% line do not exceed the text width or a line break will occur.
%
%\begin{figure*}[!t]
%\centering
%\subfloat[Case I]{\includegraphics[width=2.5in]{box}%
%\label{fig_first_case}}
%\hfil
%\subfloat[Case II]{\includegraphics[width=2.5in]{box}%
%\label{fig_second_case}}
%\caption{Simulation results for the network.}
%\label{fig_sim}
%\end{figure*}
%
% Note that often IEEE papers with subfigures do not employ subfigure
% captions (using the optional argument to \subfloat[]), but instead will
% reference/describe all of them (a), (b), etc., within the main caption.
% Be aware that for subfig.sty to generate the (a), (b), etc., subfigure
% labels, the optional argument to \subfloat must be present. If a
% subcaption is not desired, just leave its contents blank,
% e.g., \subfloat[].


% An example of a floating table. Note that, for IEEE style tables, the
% \caption command should come BEFORE the table and, given that table
% captions serve much like titles, are usually capitalized except for words
% such as a, an, and, as, at, but, by, for, in, nor, of, on, or, the, to
% and up, which are usually not capitalized unless they are the first or
% last word of the caption. Table text will default to \footnotesize as
% the IEEE normally uses this smaller font for tables.
% The \label must come after \caption as always.
%
%\begin{table}[!t]
%% increase table row spacing, adjust to taste
%\renewcommand{\arraystretch}{1.3}
% if using array.sty, it might be a good idea to tweak the value of
% \extrarowheight as needed to properly center the text within the cells
%\caption{An Example of a Table}
%\label{table_example}
%\centering
%% Some packages, such as MDW tools, offer better commands for making tables
%% than the plain LaTeX2e tabular which is used here.
%\begin{tabular}{|c||c|}
%\hline
%One & Two\\
%\hline
%Three & Four\\
%\hline
%\end{tabular}
%\end{table}


% Note that the IEEE does not put floats in the very first column
% - or typically anywhere on the first page for that matter. Also,
% in-text middle ("here") positioning is typically not used, but it
% is allowed and encouraged for Computer Society conferences (but
% not Computer Society journals). Most IEEE journals/conferences use
% top floats exclusively. 
% Note that, LaTeX2e, unlike IEEE journals/conferences, places
% footnotes above bottom floats. This can be corrected via the
% \fnbelowfloat command of the stfloats package.


\section{Introduction}
Traditional code and document inspection in terms of discovery and comprehension of mistakes and bugs is one of the most important processes in software development.
In addition to the intuitive approach of manually discovering mistakes, side effects and bugs in software and documents, there are various techniques to further enhance code and documentation quality, ranging from spell-checking, static and dynamic code analyses, automatic code quality monitoring to project check-in policies.
These techniques have one thing in common. They can mostly be applied (semi-)automatically without the need of interacting with other fellow developers.

Another technique for improving code quality, that utilizes the ‘\textit{many eyes see better than two}’ principle are review meetings.
These developer meetings generally take place after each developer independently reviewed the code or document in question.
The idea of these meetings is that all independently discovered issues get accumulated, presented and discussed in a public manner. 
Earlier work on code review meetings has shown that such review meetings are useful for discovering potential bugs, improving code quality and enforcing standards (software reviews: the state of the practice).
The question that yet remains to be answered is if code review meetings are an efficient way of improving code quality in general.
\section{Related Work}
In this section we discuss related work with respect to the efficiency of technical review meetings and the relevance of the meeting sessions. Prior work has qualitatively analyzed the technical review process used by large software systems.

\subsection{Review without meeting session}
Without a review meeting the reviewers would just send their list of findings via tools developed for supporting code review to the author of the artifact or choose one reviewer to take care of all the review work. The following papers show the influence on the quality of the artifact when using this method. \\* 
Baker discussed a procedure for reviewing code changes that are made to a software product as it moves through its life cycle. Facing a decline of customer confidence in his product his team came up with the idea of using a revision control system to identify all changes made to the product and a line by line review of the changes by the project manager. He noted that this methodology came with several advantages one of them being the efficiency with which the quality improvement was made. A single reviewer becomes very efficient in his reviews due to practice at reviewing code and to a thorough understanding of the product. The fact they used the product manager as the single reviewer facilitated him getting an insight into the whole project encompassing each engineer's performance, the complexity of the tasks, progress of the project and also bettering his scheduling skills. The employees liked this methodology because of the manager getting an understanding of the effort in their work, capabilities and identifying areas that needed improvement. \\*
By using this methodology a 40\% reduction in the number of bugs at a cost of 1-2\% of the project was achieved \cite{Baker:1997:CRE:253228.253461}. \\*
Mcintosh et al. studied the relationship between software quality and:
\begin{itemize}
	\item the proportion of changes that have been code reviewed (code review coverage), and
	\item the degree of reviewer involvement in the code review process (code review participation)
\end{itemize}
through a case study of the large Qt, VTK, and ITK open source systems. The Gerrit-driven code review process for git-based software projects has been used which is tightly integrated with test automation and code integration tools. They extracted code review data from the Gerrit review databases of the studied systems and linked the review data to the integrated patches recorded in the corresponding version control systems. They built Multiple Linear Regression (MLR) models to explain the incidence of post-release defects detected in the components of the studied systems having the goal to understand the relationship between the explanatory variables (code review coverage and participation) and the dependent variable (post-release defect counts). \\*
They found out that low code review coverage and participation are estimated to produce components with up to two and five additional post-release defects respectively \cite{McIntosh:2014:ICR:2597073.2597076}. \\*
Bacchelli et al. empirically explored the motivations, challenges, and outcomes of tool-based code reviews. They observed, interviewed, and surveyed developers and managers and manually classified hundreds of review comments across diverse teams at Microsoft using discussions within the review tool CodeFlow. They found out that the main motivation for reviews despite finding defects are knowledge transfer, increasing team awareness, and creation of alternative solutions to problems \cite{Bacchelli:2013:EOC:2486788.2486882}. \\*
Meyer discussed the importance of code review meetings in the age of the internet saying that new collaboration tools allow geographically distributed software-development teams to boost the venerable concept of code review. Seeing it almost as an impossibility to have all the people involved at a same time in the same room they used several communication tools to allow several threads of discussion to cover code, design and specification as well. Criticizing the traditional code review meeting process by saying that in an extreme programming point reviews may be superfluous in projects practicing pair programming, also when it comes to finding code flaws static analysis tools being more effective than human inspection and that the meeting process is time consuming, he still acknowledges that code reviews remain an important tool when adapted to the modern world of software development by including abstract program interface (API) design, architecture choices, and other specification and design issues to the review meetings. Another point they make is that a physical meeting among people in the same room is becoming hardly applicable since distributed teams spread over many locations and time zones and that distributed reviews are possible with the right tools. They used a combination of tools including X-Lite, a voice communication similar to a conference call, Skype for written communication, Google Docs for shared documents and WebEx, a sharing tool for sharing screens. This brought many advantages one of them being saving time by updating the document in Google Docs in real time during the meeting, preparing the shared document with links to the actual code a week ahead of the review and providing their comments on the review page (the shared document) prior to the meeting with the code author then responding just below the comments on the same page \cite{Meyer_2008}. \\*
McCarthy et al. designed and conducted a controlled experiment in the spring of 1995 with 21 subjects acting as reviewers with the goals being to characterize the behaviour of existing approaches, and to assess the potential benefits of meetingless inspections. The three inspection methods were:
\begin{itemize}
	\item every reviewer individually analyzes the artifact with no detecting defects but preparing it for the review meeting (PI)
	\item each reviewer analyzes the artifact with detecting as many defects as possible, inspecting the document in a review meeting (DC)
	\item each reviewer analyzes the artifact with the goal of detecting as many defects as possible, then again repeat this procedure a second time with not review meeting (DD)
\end{itemize}
The experiment manipulated four independent variables:
\begin{itemize}
	\item The inspection method (PI, DC or DD)
	\item The inspection round (each reviewer participated in two inspections during the experiment)
	\item The specification to be inspected (two were used during the experiment)
	\item The order in which the specifications were inspected (Either specification could be inspected first)
\end{itemize}
For each inspection they measured three dependent variables:
\begin{itemize}
	\item The individual defect detection rate
	\item The team defect detection rate
	\item The gain rate, that is, the percentage of defects initially identified during the second phase of the inspection
\end{itemize}
They found out that the inspection method used cannot be ignored as a significant source of variation in the meeting gain rates and that the meetingless inspections detected more new defects in the second phase of the inspection than did inspections using the other methods and that overall more defects were found than in inspections with meetings \cite{mccarthy1996experiment}. \\*
Votta determined the proportion of defects found during the inspection that were originally discovered at the meeting (meeting gain rate) to quantify the usefulness of inspection meetings in a case study of 13 design inspections at the American Telephone and Telegraph Company (AT\&T). The result was that the meeting gain rate for these inspections was ~5\%, meaning that review meetings are not effective at all \cite{votta1993does}. \\*
Porter et al. did another study collecting data on meeting gains, also at AT\&T involving >100 code inspections observing many meetings producing no gains at all, while some having rates of 80\% with an average meeting gain rate of 33\% \cite{porter1995experiment}. \\*
Porter et al. reported that although the effectiveness of software inspection is valid their economics remain uncertain figuring out that the benefits of inspections are often overstated and the costs, especially large software development projects, are understated pointing to the fact that some of the most influential studies establishing these costs and benefits are 20 years old now \cite{porter1996review}.

\section{Experimental Design}

\subsection{Research Questions}
First, we wanted to see, whether review meetings help the software development process in terms of quality management.

\textbf{RQ1: Are review meetings effective in regards to improving document or code quality?}

Next, we were interested in how efficient these review meetings are in regards to time and monetary requirements.

\textbf{RQ2: How efficient are code review meetings as a means to increasing document or code quality?}

Additionally to direct improvements to the code via the review reports, other studies concerning the topic suggest, that the execution of reviews may provide additional benefits such as knowledge transfer, increased team awareness, or creation of alternate solutions \cite{expecations...}.

\textbf{RQ3: Which positive effects besides improving code quality do review meetings entail?}

%\textit{RQ3:} Which requirements have to be met in order to have effective %review meetings?

%\textit{RQ4:} Which requirements have to be met in order to have efficient %review meetings?

\subsection{Hypotheses}

Initially, we defined three hypotheses and their according null hypotheses.
The first hypothesis refers to RQ1. We wanted to understand if review meetings help improving document quality at all and thus only look at the effectiveness.

\textbf{H1: Review meetings are effective in improving document quality.}

The respective null hypothesis assumes that there is no significant improvement in document quality.

\textbf{H10: Review meetings are not effective in improving document quality.}

Next we were interested in whether or not review meetings are also efficient in improving document quality, since a group of usually five or six software developers is required and thus entails high monetary costs.

\textbf{H2: Review meetings are an efficient way of increasing document quality.}

The null hypothesis for H2 presumes that review meetings are not efficient in improving document quality.

\textbf{H20: Review meetings are not an efficient way of increasing document quality.}

Finally, we defined our third hypothesis concerning the research question of whether or not review meetings bring positive side effects to the table.

\textbf{H3: Review meetings have a positive impact besides the direct improvement to document quality.}

The corresponding null hypothesis, assuming that there are no positive side effects, looks as follows:

\textbf{H30: Review meetings do not have any positive impact besides the direct improvement to document quality.}

\subsection{Design}

Quantitative and qualitative analysis.?
One factor 2 treatment.

For our quantitative analysis we used a one factor two treatment design. The first treatment is the control treatment of dong no review meetings, while the second treatment is the execution of a review meeting. Since doing no review does not influence the subjects, each group was assigned to both treatments.

\definecolor{heading}{RGB}{200,200,255}
\definecolor{a}{RGB}{220,220,255}
\definecolor{b}{RGB}{230,230,255}

\begin{table}
\centering
\begin{tabular}{lcr}
  \rowcolor{heading}Subjects & No Review Meeting & Review Meeting \\
  \rowcolor{a}"SoPra" Group 1 & x & x \\
  \rowcolor{b}"SoPra" Group 2 & x & x \\
  \rowcolor{a}... & x & x \\
\end{tabular}
\caption{The one factor two treatment design used for our experiment.}
\end{table}


\subsection{Objects}

The idea of our experiment was that we analyse the review report that result out of the review meetings and compare them to the merged lists of findings from each individual reviewer.

\subsubsection{Review Meetings}

The review meetings are conducted by students of the University of Stuttgart within the scope of the "Software Praktikum". Each review group consists of five people, being three reviewers, one moderator and one scribe, who is also representing the authors. Before the review meeting itself, all reviewers have to inspect the document and create a list of findings using a review tool \cite{TODO:revager Thommy} tool. The reviews are expected to have a duration of TODO:90 minutes. During the reviews, the scribe also uses a review meeting tool to gather the findings and in the end creates a final review report.

\subsubsection{Review Tools}

We looked into the following review tools:
\textit{RevAger:}
\textit{Collaborator}
TODO Thommy: Find more review tools, explain them, decide for RevAger (and find a reason).

\subsection{Experiment Procedure}
TODO: Josip

\subsection{Data collection Procedure}

The data for the control treatment is gathered by taking the findings lists of each reviewer in a review group and merging these lists together, eliminating duplicates.
On the other hand, the data for the review treatment equals the findings list that results out of the review. Additionally, the RevAger tool automatically measures the elapsed time.
The tool also saves the participants and their respective roles.

The structure of a finding consists of a description, an aspect (what reviewers should have a special focus on, e.g. completeness), a reference (where to find it in the document) and an importance rating.

TODO Haris: qualitative data collection


\subsection{Analysis Procedure}

For the analysis of the findings, we decided to evaluate them manually, due to the fact that there is no objective way of rating the quality of a finding. The evaluation will be executed by three of the study conductors independently and the final result is determined by establishing the median of these.

In order to rate the individual findings, we constructed the following classification scale, which can also be seen in table~\ref{tab:ratings}:
The weighing of the findings are worth 1 for good, 2 for besides error, 3 for main error and 5 for crucial error. If a rating is incorrect, we will instead assign the value for the correct weighting and decrease it by 1.
If a finding itself is incorrect (e.g. claims to have found a mistake, where there is none), the finding will recieve a rating of -1.

Next, we tried to find formula that determines the overall quality of a finding list. It is important to consider both the amount of findings, but also the overall quality of these findings. Altough more findings generally are a positive result, we decided that the overall quality is even more important (e.g. to reduce stacking of rather unimportant "good" findings). Therefore our focus is set more on the overall finding quality, meaning that we rate critical and main errors higher.
We defined the amount of findings $a_L$ by:
\begin{center}

$a_L = \sum_{L}$
\end{center}

Where $L$ is the merged list of all findings, without duplicates.

For the overall quality $q_L$ we determined the arithmetic mean of the accumulated weightings of the findings.

\begin{center}
{$q_L = \frac{1}{a_L} \sum_{f \in L} w_f$}
\end{center}

Where $w_f$ is the weighting (see table~\ref{tab:ratings}) of finding $f$.

In order to evaluate the difference between two finding lists, we defined the following function:

\begin{center}
$f(L_1, L_2) = \frac{a_{L_2} - a_{L_1}}{a_{L_2} + a_{L_1}} + (q_{L_2} - q_{L_1}) $
\end{center}

While in theory $f \in [-7, 7]$ holds with $\frac{a_{L_2} - a_{L_1}}{a_{L_2}} \in [-1, 1]$ and $q_{L_2} - q_{L_1} \in [-6, 6]$, in reality the values we will recieve will be very small. Suppose the reviewers assemble 100 unique findings beforehand and the review meeting results in 5 new findings (which would be a unusually high amount), our first term
would have a value of 0.0244.
As for the second term, suppose the average rating for the findings was 3.0 and 10 classification errors occured, we would receive a final quality rating $q_{L_1}$ of 2.951. Assuming that the review meeting corrects all these classification errors, the second term would add up to 0.049. So overall our improvement is 0.0734. Assuming these values are realistic, our previously mentioned focus on the quality would be given with approximately 2:1.
Considering these values, we now can define our hypotheses.
We define effectiveness as having any improvement from the document without the meeting.
We define our H1 Hypothesis on effectiveness as following:

\begin{center}
\textbf{H1:} $f(L_{pre}, L_{post}) > 0$
\end{center}

Wheres $L_pre$ is the finding list before the review meeting, while $L_post$ describes the finding list that results after the meeting.
The corresponding null hypothesis, assuming that there is no improvement resulting out of a review meeting, looks as follows:

\begin{center}
\textbf{H10:} $f(L_{pre}, L_{post}) \le 0$
\end{center}

As for efficiency, we first have to consider the time and monetary costs that result out of a review meeting. Assuming that a review consists of three reviewers, one moderator, one scribe and an author with an average earnings of 100€/h per developer, an hour would cost approximately 600€, not considering potential costs of the meeting room. Since this research only focuses on the review meetings themselves, we do not consider any costs that arise within a review process outside the meeting itself (e.g. the cost for the reviewers to analyse the specimen). Estimating the cost of an error poses a difficult task. According to \cite{stecklein2004error} the cost of an error in the requirements phase increases by 30 to 70 times, when found in the acceptance testing, where a specification error is most likely found in. We assume that correcting an average error in the specification takes about 3 minutes to correct and thus has an initial cost of about 5€, which will cost about 250€ when found in the acceptance testing.
We assume that this is the average cost for a critical error in order to not overestimate the efficiency. Further we assume that a besides error is worth 100€ and a main error is worth 150€. We define our hypothesis H2, that describes the efficiency of review meetings, as follows:

\begin{center}
\textbf{H2:} $100 * n < 50* \sum_{f \in L_{new}} w_f$
\end{center}

Where $n$ is the amount of developers taking part in the review meeting. $L_{new}$ is the list of newly discovered findings, defined as $L_{new} = L_{post} \setminus L_{pre}$.

\begin{table}
\centering
\begin{tabular}{lr}
  \rowcolor{heading}Classification & Rating \\
  \rowcolor{a}'Good' & 1 \\
  \rowcolor{b}'Besides Error' & 2 \\
  \rowcolor{a}'Main Error' & 3 \\
  \rowcolor{b}'Critical Error' & 5 \\
  \rowcolor{a}'Finding Error' & -1 \\
  \rowcolor{b}'Classification Error Penalty' & -1 \\
\end{tabular}
\caption{Rating table for the evaluation of findings.}
\label{tab:ratings}
\end{table}

\subsection{Validity Procedure}

The review teams were assigned semi-randomly out of all "SoPra" participants with only one condition, being that no "SoPra" team members are within the same review group.

In order to reduce subjectivity of the manual evaluation that will take place when weighting the findings, we decided that three study conductors independently rate these weightings. The median of these three ratings will be taken as the final value.

Each group recieved the specification document three days before their respective review meetings. By doing this we strived for having their memories on the specimen as fresh as possible.
%\section{Analysis}
\section{Threats to Validity}

\subsection{Conclusion Validity}

The threat level to conclusion validity of our experiment is rather high. This is due to the fact
that some of our measurements could not be automatically calculated but instead had to be determined
manually by the conductors. In order to reduce this threat, we decided to let three conductors
determine the values independently and finally took the median as our final result.
Also since the specification was created by us, we are biased when evaluating the findings. This threat can be eliminated by separating the persons doing the specification and the persons evaluating the findings.
Our calculations on the efficiency hypothesis H2 are mainly based on money estimates, and the costs of developers varies between countries, companies and individual position within a company.

\subsection{Internal Validity}

Due to the fact that the participants had to be informed about the experiment to give us their
consent, they knew that some part of their work will be evaluated and thus might have been
influenced in their behavior by the experiment, e.g. resulting in them making more efforts than they would have done otherwise.

Since we cannot prevent students from communicating with each other, it is unavoidable that information about the specification or experiment in general could be exchanged.

\subsection{Construct Validity}

The construct validity between the treatment and the cause construct is a given, since they are identical, being the review meeting.

As for the validity between the outcome and the construct of the effect, ?

\subsection{External Validity}

Our study also contains external threats to validity. Since every group of students in the "SoPra"
has to solve the exact same task, all produces code or specifications should ideally be semantically
equal and thus our results might not necessarily be generalizable to all kind of software systems.

Another external factor that threatens validity are the participants of the study. These are limited
to students of the university Stuttgart and thus represent only a minor part of software engineers.

Lastly, the size of the "SoPra" is very small, since it is only a six month project for three developers and thus is far off from realistic industrial projects.


\section{Results and Analysis}

\begin{table*}[t]
\centering
\begin{tabular}{c|cccccccccccccc}
  Group No. &  \multicolumn{2}{r}{Good Error} & \multicolumn{2}{r}{Besides Error} & \multicolumn{2}{r}{Main Error} & \multicolumn{2}{r}{Critical Error} & \multicolumn{2}{r}{Finding Error} & \multicolumn{2}{r}{Classification Error} & Time & Developer\\
  & Pre & Post & Pre & Post & Pre & Post & Pre & Post & Pre & Post & Pre & Post & & Amount  \\
  \hline
  Group 1 &0&0&23&23&8&9&3&3&1&1&1&1&63&6 \\
  Group 2 &0&0&27&29&10&10&4&4&2&1&1&0&67&6 \\
  Group 3 &0&0&30&30&11&11&4&4&0&0&0&0&65&6 \\
  Group 4 &0&0&40&40&12&12&5&5&1&0&0&0&81&6 \\
  Group 5 &0&0&35&37&9&10&4&4&2&0&1&0&76&5 \\
  Group 6 &0&0&46&45&10&9&4&5&2&1&1&0&79&6 \\
  Group 7 &0&0&47&47&13&12&5&5&1&0&0&0&87&6 \\
  Group 8 &0&0&34&35&9&9&4&4&1&1&0&0&62&6 \\
  Group 9 &0&0&22&22&8&7&3&4&2&1&2&1&58&6 \\
  Group 10 &0&0&33&35&11&11&4&4&0&0&0&0&74&6 \\
  Group 11 &0&0&39&39&10&10&4&4&1&1&0&0&67&6 \\
  Group 12 &0&0&25&25&7&8&3&3&2&1&1&0&62&6 \\
  Group 13 &0&0&27&27&8&8&3&4&0&0&2&0&58&5 \\
  Group 14 &0&0&34&35&11&11&4&4&1&0&0&0&65&6 \\
  Group 15 &0&0&48&48&12&12&5&5&0&0&0&0&80&6 \\
\end{tabular}
\caption{Mock data of the quantitative data collection.}
\label{tab:results}
\end{table*}

Since we did not conduct the actual experiment, no real data is present. Table~\ref{tab:results} shows mock data to demonstrate what the real data should look like.
\section{Conclusion and Future Work}

Focusing on the efficiency of review meetings we covered an essential part regarding the document review meetings and whether they have an influence on enhancing code quality.  

As we only focus on document review meetings a possible future work is to conduct a similar experiment study taking a closer look at code review meetings. The results can further be compared to studys, where only commonly used code inspection techiques are used, such as static or dynamic code inspections, and draw conclusions whether it is usefull to aditionally conduct code review meetings.   

To have a better understanding of how many costs are involved within a review meeting it is also important to cunduct a study to detect e.g. how much a bug costs during the various stages of software developememt, as well as how much it costs to engage on developer in this process etc. With this knowledge a better understanding of the importance of review meetings can be achieved.sdfsdf
sdsdf     

% conference papers do not normally have an appendix


% use section* for acknowledgment
\section*{Acknowledgment}


The authors would like to thank...





% trigger a \newpage just before the given reference
% number - used to balance the columns on the last page
% adjust value as needed - may need to be readjusted if
% the document is modified later
%\IEEEtriggeratref{8}
% The "triggered" command can be changed if desired:
%\IEEEtriggercmd{\enlargethispage{-5in}}

% references section

% can use a bibliography generated by BibTeX as a .bbl file
% BibTeX documentation can be easily obtained at:
% http://mirror.ctan.org/biblio/bibtex/contrib/doc/
% The IEEEtran BibTeX style support page is at:
% http://www.michaelshell.org/tex/ieeetran/bibtex/
\bibliographystyle{IEEEtran}
% argument is your BibTeX string definitions and bibliography database(s)
\bibliography{IEEEabrv,../IEEEabrv}
%
% <OR> manually copy in the resultant .bbl file
% set second argument of \begin to the number of references
% (used to reserve space for the reference number labels box)
%\begin{thebibliography}{1}

%\bibitem{IEEEhowto:kopka}
%H.~Kopka and P.~W. Daly, \emph{A Guide to \LaTeX}, 3rd~ed.\hskip 1em plus
%  0.5em minus 0.4em\relax Harlow, England: Addison-Wesley, 1999.

%\end{thebibliography}




% that's all folks
\end{document}


